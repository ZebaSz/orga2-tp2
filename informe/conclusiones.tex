\section{Conclusiones}
	A grandes rasgos, podemos concluir que el uso de operaciones SIMD es altamente beneficioso en muchos contextos dentro del procesamiento de imágenes. Si bien existen casos donde su uso no aporta grandes benficios (como pueden ser los casos de los conversores RGB-YUV o FourCombine), en la mayoría de los filtros la diferencia de performance era notoria.

	Cabe destacar que la implementación de los algoritmos en ASM con operaciones SIMD fue mucho más costosa que la versión correspondiente en lenguajes de alto nivel, y de no ser realizada con ciudado puede llevar no a mejora sino a pérdida de performance. En particular, en una implementación inicial de uno de los filtros hallamos información que estaba siendo calculada más de una vez, anulando los beneficios de la paralelización. Incluso dentro de nuestra experimentación con LinearZoom podemos ver que obtener la implementación más eficiente posible no es trivial.

	Los casos donde consideramos más útil la implementación con SIMD es en aquellos que utilizan un kernel de píxeles (como en MaxCloser). En dichos filtros cada pixel debe procesarse de manera distinta múltiples veces (para encontrar su reemplazo y como parte de un kernel), y resulta difícil (o imposible) reutilizar valores intermedios.