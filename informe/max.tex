\section{Maximo cercano}

Este es un filtro en el que para cada pixel de la imagen original, se genera un pixel nuevo buscando cual es el máximo valor de cada componente en los píxeles de alrededor, y con este nuevo pixel hacemos una combinación lineal con el original, y un parámetro que nos dan y lo ponemos en la imagen destino en la misma posición que el original. La búsqueda de las componentes máximas, se hace sobre un kernel de 7x7 píxeles centrado en el pixel que estamos interesados en cambiar.

\begin{table}[h]
\centering
\begin{tabular}{l|c|c|c|c|c|c|c|l}
 & \multicolumn{1}{l|}{}       & \multicolumn{1}{l|}{}      & \multicolumn{1}{l|}{}       & \multicolumn{1}{l|}{}       & \multicolumn{1}{l|}{}       & \multicolumn{1}{l|}{}       & \multicolumn{1}{l|}{}      &  \\ \hline
 & \cellcolor[HTML]{FAFE8E}$P_{00}$ & \cellcolor[HTML]{FAFE8E}$P_{01}$  & \cellcolor[HTML]{FAFE8E}$P_{02}$  & \cellcolor[HTML]{FAFE8E}$P_{03}$  & \cellcolor[HTML]{FAFE8E}$P_{04}$  & \cellcolor[HTML]{FAFE8E}$P_{05}$ & \cellcolor[HTML]{FAFE8E}$P_{06}$ &  \\ \hline
 & \cellcolor[HTML]{FAFE8E}$P_{10}$ & \cellcolor[HTML]{FAFE8E}$P_{11}$  & \cellcolor[HTML]{FAFE8E}$P_{12}$  & \cellcolor[HTML]{FAFE8E}$P_{13}$  & \cellcolor[HTML]{FAFE8E}$P_{14}$  & \cellcolor[HTML]{FAFE8E}$P_{15}$ & \cellcolor[HTML]{FAFE8E}$P_{16}$ &  \\ \hline
 & \cellcolor[HTML]{FAFE8E}$P_{20}$ & \cellcolor[HTML]{FAFE8E}$P_{21}$  & \cellcolor[HTML]{FAFE8E}$P_{22}$  & \cellcolor[HTML]{FAFE8E}$P_{23}$  & \cellcolor[HTML]{FAFE8E}$P_{24}$  & \cellcolor[HTML]{FAFE8E}$P_{25}$ & \cellcolor[HTML]{FAFE8E}$P_{26}$ &  \\ \hline
 & \cellcolor[HTML]{FAFE8E}$P_{30}$ & \cellcolor[HTML]{FAFE8E}$P_{31}$  & \cellcolor[HTML]{FAFE8E}$P_{32}$  & \cellcolor[HTML]{FE8E8E}$P_{33}$  & \cellcolor[HTML]{FAFE8E}$P_{34}$  & \cellcolor[HTML]{FAFE8E}$P_{35}$ & \cellcolor[HTML]{FAFE8E}$P_{36}$ &  \\ \hline
 & \cellcolor[HTML]{FAFE8E}$P_{40}$ & \cellcolor[HTML]{FAFE8E}$P_{41}$  & \cellcolor[HTML]{FAFE8E}$P_{42}$  & \cellcolor[HTML]{FAFE8E}$P_{43}$  & \cellcolor[HTML]{FAFE8E}$P_{44}$  & \cellcolor[HTML]{FAFE8E}$P_{45}$ & \cellcolor[HTML]{FAFE8E}$P_{46}$ &  \\ \hline
 & \cellcolor[HTML]{FAFE8E}$P_{50}$ & \cellcolor[HTML]{FAFE8E}$P_{51}$  & \cellcolor[HTML]{FAFE8E}$P_{52}$  & \cellcolor[HTML]{FAFE8E}$P_{53}$  & \cellcolor[HTML]{FAFE8E}$P_{54}$  & \cellcolor[HTML]{FAFE8E}$P_{55}$ & \cellcolor[HTML]{FAFE8E}$P_{56}$ &  \\ \hline
 & \cellcolor[HTML]{FAFE8E}$P_{60}$ & \cellcolor[HTML]{FAFE8E}$P_{61}$  & \cellcolor[HTML]{FAFE8E}$P_{62}$  & \cellcolor[HTML]{FAFE8E}$P_{63}$  & \cellcolor[HTML]{FAFE8E}$P_{64}$  & \cellcolor[HTML]{FAFE8E}$P_{65}$ & \cellcolor[HTML]{FAFE8E}$P_{66}$ &  \\ \hline
 & \multicolumn{1}{l|}{}      & \multicolumn{1}{l|}{}      & \multicolumn{1}{l|}{}       & \multicolumn{1}{l|}{}       & \multicolumn{1}{l|}{}       & \multicolumn{1}{l|}{}       & \multicolumn{1}{l|}{}      &
\end{tabular}
\caption{En rojo el pixel que estamos editando y en amarillo los píxeles que estan dentro del kernel}
\end{table}


\subsection{Implementación}

Para la implementación de este filtro, recorremos la imagen original, iterando sobre sus filas y sus columnas, como hay píxeles que no tenemos un kernel de 7x7 alrededor, estos los pintamos de blanco, pero si podemos, iteramos sobre el kernel y nos vamos fijando cuáles son las componentes máximas y cuando recorrimos todo el kernel, para cada componente hacemos esta cuenta: \\ Componente Destino $\leftarrow$ Componente Original * (1 - VAL) + Componente Máxima * VAL. (Donde VAL es el parámetro que nos pasan en la función).

Al implementar este filtro en lenguaje ensamblador, podemos aprovechar de las ventajas que nos brinda el modelo SIMD. En particular, los registros XMM son de 16 bytes, que los podemos utilizar para procesar 4 píxeles en paralelo. Para esta implementación, vamos a aprovechar estos registros para buscar el máximo sobre el kernel y hacer la combinación lineal sobre cada componente en paralelo.

Como dijimos recorremos las columnas y filas, primero nos fijamos si es una fila que tenemos que pintar de blanco, en caso afirmativo, sabemos que toda esa fila va a hacer blanca, entonces podemos aprovechar los registros XMM para guardar en memoria múltiples pixeles y como nos entran 4, podemos generarnos un registro XMM que contenga 4 píxeles blancos y guardarlos en memoria en la imagen destino a la vez. 

Ahora sí es un píxel que debemos calcularlo, tenemos que iterar sobre el kernel y buscar el máximo de cada color. Lo que podemos hacer para aprovechar SIMD, es cargar 4 píxeles en un registro y otros 4 en otro, ahora si aplicamos la instrucción PMAXUB, que compara byte a byte entre los registros y guarda el máximo de los dos en el destino. Entonces lo que ganamos con esto es que en el registro destino nos quedó 4 píxeles que cada tiene el máximo de cada componente entre el pixel que estaban en la misma posición de los 2 registros. 

\begin{center}
\xmm{1} \xmmDoubleWordSmall{$P_1$}{$P_2$}{$P_3$}{$P_4$} \\
\xmm{2} \xmmDoubleWordSmall{$P_5$}{$P_6$}{$P_7$}{$P_8$}

\texttt{PMAXUB} \xmm{1}, \xmm{2} \hfill

\xmm{1} \xmmDoubleWordSmall{\tiny$MAX(P_1,P_5)$}{\tiny$MAX(P_2,P_6)$}{\tiny$MAX(P_3,P_7)$}{\tiny$MAX(P_4,P_8)$} \\
\end{center}

$MAX(PM,P)$ = \xmmDoubleWordSmall{\tiny$MAX(PM^r,P^r)$}{\tiny$MAX(PM^g,P^g)$}{\tiny$MAX(PM^b,P^b)$}{\tiny$MAX(PM^a,P^a)$} \\

Usando esta metodología podemos mantener un XMM que contenga los píxeles máximos y lo comparamos contra los del kernel, actualizando este registro. Pero como cada fila del kernel tiene 7 píxeles contiguos, hacemos esta técnica dos veces, pero estaríamos comparando 8 pixeles, así que repetimos un píxel en cada paso asi comparamos todos los píxeles. Hacemos esto para cada fila del kernel y nos termina quedando un registro con 4 posibles máximos, luego debemos compararlos entre sí. Para ello copiamos el registro donde tenemos los posibles máximos, shifteamos a la derecha uno de ellos, 8 bytes shifteamos para que nos quede desplazado 2 pixeles. Y volvemos a comparar, ahora nos quedan solamente 2 y de vuelta desplazamos los pixeles pero ahora solamente 4 bytes, y comparamos de vuelta, quedandonos el píxel máximo.

\begin{center}
\xmm{1} \xmmDoubleWordSmall{$M_1$}{$M_2$}{$M_3$}{$M_4$} \\
\xmm{2} $\leftarrow$ \xmm{1} \\
\texttt{PSRLDQ} \xmm{2}, \texttt{8} \hfill \\
\xmm{2} \xmmDoubleWordSmall{0}{0}{$M_1$}{$M_2$} \\

\texttt{PMAXUB} \xmm{1}, \xmm{2} \hfill

\xmm{1} \xmmDoubleWordSmall{$M_1$}{$M_2$}{\tiny$MAX(M_1,M_3)$}{\tiny$MAX(M_2,M_4)$} \\

\xmm{2} $\leftarrow$ \xmm{1} \\
\texttt{PSRLDQ} \xmm{2}, \texttt{4} \hfill \\
\xmm{2} \xmmDoubleWordSmall{0}{$M_1$}{$M_2$}{\tiny$MAX(M_1,M_3)$} \\

\texttt{PMAXUB} \xmm{1}, \xmm{2} \hfill \\

\xmm{1} \vspace{0.1cm}
\begin{tabular}{|C{1cm}|C{1cm}|C{1cm}|C{3.8cm}|}\hline
X & X & X & $MAX(M_1,M_2,M_3,M_4)$ \\ \hline
\end{tabular}
\vspace{0.1cm}

\end{center}

Ya conseguido el máximo, ahora tenemos que realizar la combinación lineal con el píxel original, el nuevo píxel que generamos y el parámetro. Lo que queremos es hacer la cuenta en paralelo, para eso, podemos usar instrucciones SIMD para multiplicar las componentes de los píxeles en paralelo por el parámetro, que es un float, entonces convertimos a float cada componente quedándonos todas las componentes en un registro XMM, y para multiplicarlas por el parámetro tenemos que hacer un registro XMM que lo contenga 4 veces y aplicamos MULPS, quedando de resultado la multiplicación por cada componente contra el parámetro cada una en floats. Luego hacemos lo mismo con el píxel original, y sumamos estos dos resultados. Ahora nos queda guardar este resultado en la imagen destino, pero antes, debemos convertir los floats a byte.s

\subsection{Análisis preeliminar}
\subsubsection*{Comparación de rendimiento de ASM vs C}
\subsubsection*{Comparar para distintos tamaños, relaciones entre implementaciones}

\subsection{Hipótesis de trabajo}
\subsubsection*{Conjunto de ideas de experimentos}
\subsubsection*{Afirmaciones que buscan probar verdaderas}
\subsubsection*{Deben ser concisas y claras}

\subsection{Diseño experimental}
\subsubsection*{Explicación de como y que van a medir}
\subsubsection*{Explicación del conjunto de datos de entrada}
\subsubsection*{Detalles de la plataforma y la configuración de la misma}

\subsection{Resultados y Análisis}
\subsubsection*{Resultados obtenidos, gráficos y tablas}
\subsubsection*{Explicación e interpretación de los resultados obtenidos}

\subsection{Conclusiones}
\subsubsection*{Relación entre las hipótesis de trabajo y resultados}