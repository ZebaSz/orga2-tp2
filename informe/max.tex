\section{Maximo cercano}

Este es un filtro tal que cada pixel de la imagen original se edita de la siguiente manera:

\begin{enumerate}
	\item Obtenemos los pixeles de alrededor, que se encuentren a una distancia de 3 pixeles o menos. A esto lo denominamos kernel, en este caso, de 7x7.

	\item Se calcula el máximo valor para cada componente entre estos pixeles obtenidos.

	\item Se genera un nuevo pixel con estos 3 valores encontrados.

	\item El pixel final se calcula mediante una combinación lineal entre el pixel original y el pixel que contiene las componentes máximas.

	\item Aquellos pixeles para los cuales no se puede armar un kernel de 7x7 alrededor (las primeras y últimas 3 columas y 3 filas de la imagen) deben ser pintados de blanco.
\end{enumerate}

La combinación lineal entre los pixeles se hace de la siguiente manera:

\begin{center}
	$src * (1 - val) + max * (val)$
\end{center}

Donde $src$ es el pixel original, $max$ es el pixel generado, y $val$ es un número flotante entre 0 y 1, este mismo es un parámetro del filtro.

\begin{table}[h]
\centering
\begin{tabular}{l|c|c|c|c|c|c|c|l}
 & \multicolumn{1}{l|}{}       & \multicolumn{1}{l|}{}      & \multicolumn{1}{l|}{}       & \multicolumn{1}{l|}{}       & \multicolumn{1}{l|}{}       & \multicolumn{1}{l|}{}       & \multicolumn{1}{l|}{}      &  \\ \hline
 & \cellcolor[HTML]{FAFE8E}$P_{00}$ & \cellcolor[HTML]{FAFE8E}$P_{01}$  & \cellcolor[HTML]{FAFE8E}$P_{02}$  & \cellcolor[HTML]{FAFE8E}$P_{03}$  & \cellcolor[HTML]{FAFE8E}$P_{04}$  & \cellcolor[HTML]{FAFE8E}$P_{05}$ & \cellcolor[HTML]{FAFE8E}$P_{06}$ &  \\ \hline
 & \cellcolor[HTML]{FAFE8E}$P_{10}$ & \cellcolor[HTML]{FAFE8E}$P_{11}$  & \cellcolor[HTML]{FAFE8E}$P_{12}$  & \cellcolor[HTML]{FAFE8E}$P_{13}$  & \cellcolor[HTML]{FAFE8E}$P_{14}$  & \cellcolor[HTML]{FAFE8E}$P_{15}$ & \cellcolor[HTML]{FAFE8E}$P_{16}$ &  \\ \hline
 & \cellcolor[HTML]{FAFE8E}$P_{20}$ & \cellcolor[HTML]{FAFE8E}$P_{21}$  & \cellcolor[HTML]{FAFE8E}$P_{22}$  & \cellcolor[HTML]{FAFE8E}$P_{23}$  & \cellcolor[HTML]{FAFE8E}$P_{24}$  & \cellcolor[HTML]{FAFE8E}$P_{25}$ & \cellcolor[HTML]{FAFE8E}$P_{26}$ &  \\ \hline
 & \cellcolor[HTML]{FAFE8E}$P_{30}$ & \cellcolor[HTML]{FAFE8E}$P_{31}$  & \cellcolor[HTML]{FAFE8E}$P_{32}$  & \cellcolor[HTML]{FE8E8E}$P_{33}$  & \cellcolor[HTML]{FAFE8E}$P_{34}$  & \cellcolor[HTML]{FAFE8E}$P_{35}$ & \cellcolor[HTML]{FAFE8E}$P_{36}$ &  \\ \hline
 & \cellcolor[HTML]{FAFE8E}$P_{40}$ & \cellcolor[HTML]{FAFE8E}$P_{41}$  & \cellcolor[HTML]{FAFE8E}$P_{42}$  & \cellcolor[HTML]{FAFE8E}$P_{43}$  & \cellcolor[HTML]{FAFE8E}$P_{44}$  & \cellcolor[HTML]{FAFE8E}$P_{45}$ & \cellcolor[HTML]{FAFE8E}$P_{46}$ &  \\ \hline
 & \cellcolor[HTML]{FAFE8E}$P_{50}$ & \cellcolor[HTML]{FAFE8E}$P_{51}$  & \cellcolor[HTML]{FAFE8E}$P_{52}$  & \cellcolor[HTML]{FAFE8E}$P_{53}$  & \cellcolor[HTML]{FAFE8E}$P_{54}$  & \cellcolor[HTML]{FAFE8E}$P_{55}$ & \cellcolor[HTML]{FAFE8E}$P_{56}$ &  \\ \hline
 & \cellcolor[HTML]{FAFE8E}$P_{60}$ & \cellcolor[HTML]{FAFE8E}$P_{61}$  & \cellcolor[HTML]{FAFE8E}$P_{62}$  & \cellcolor[HTML]{FAFE8E}$P_{63}$  & \cellcolor[HTML]{FAFE8E}$P_{64}$  & \cellcolor[HTML]{FAFE8E}$P_{65}$ & \cellcolor[HTML]{FAFE8E}$P_{66}$ &  \\ \hline
 & \multicolumn{1}{l|}{}      & \multicolumn{1}{l|}{}      & \multicolumn{1}{l|}{}       & \multicolumn{1}{l|}{}       & \multicolumn{1}{l|}{}       & \multicolumn{1}{l|}{}       & \multicolumn{1}{l|}{}      &
\end{tabular}
\caption{Kernel, en rojo el pixel que estamos editando y en amarillo los pixeles que forman parte del kernel}
\end{table}


\subsection{Implementación}
 
La implementación en C se trató de hacer lo más simple posible. Iteramos las filas y las columnas, y en cada paso preguntamos si el pixel se encuentra en el margen de 3 pixeles. Si es así, lo pintamos de blanco; si no, nos generamos un pixel con cada componente inicializada en el mínimo valor(0) y recorremos el kernel y usamos este pixel que acabamos de generar para usarlo de máximo actual,  modificando cada componente cuando encontramos un valor mayor al actual. Una vez iterado el kernel realizamos la combinación lineal sobre cada componente y guardamos el pixel en la imagen destino.
 
En cambio, en la implementación del filtro en lenguaje ensamblador, esta se ve complejizada ya que podemos aprovechar de las ventajas que nos brinda el modelo SIMD. En particular, los registros XMM son de 16 bytes, que los podemos utilizar para procesar 4 pixeles en paralelo. Para esta implementación, vamos a aprovechar estos registros para buscar el máximo sobre el kernel y hacer la combinación lineal sobre cada componente en paralelo.
 
En este caso primero recorremos las filas y diferenciamos 2 casos: el caso en que toda la fila se encuentra en el margen que se pinta de blanco y el que no.
 
Si es una fila blanca, vamos a intentar guardar la mayor cantidad de pixeles contiguos posible. Como ya vimos que en los registros XMM nos caben 4, nos generamos un registro tal que los 4 pixeles sean blancos y guardarlos en memoria a la vez, al ser imágenes con ancho múltiplo de 4. Podemos aplicar esto sobre toda la fila, iterando de a 4 columnas pero vez. Con esto logramos en las primeras filas y las últimas una reducción al iterar las columnas.
 
Si es una fila que no va a pertenecer en su completitud al margen, vamos a tener que calcular el máximo de cada color en el kernel y después hacer la combinación lineal. En la iteración del kernel, en diferencia a C, podemos guardarnos en un solo registro 4 pixeles tal que puedan ser los máximos. Para esto usamos la instrucción PMAXUB, que compara byte a byte entre los registros y guarda el máximo en el registro destino. Gracias a esta instrucción podemos ir comparando estos 4 posibles máximos contra 4 pixeles contiguos del kernel e ir manteniendo los máximos parciales. Cabe destacar que la comparación es byte a byte, así que va a comparar las componentes sin desempaquetarlas. Como en el kernel tenemos nada más 7 pixeles contiguos, vamos a aplicar este método 2 veces por fila, repitiendo un pixel de la fila en cada aplicación sin que esto afecte el resultado final.

\begin{figure}[H]
	\centering
	\includegraphics{img/max/filaKernel.pdf}
	\caption{pixeles contiguos de una fila del kernel, las llaves indican que pixeles se toman en las 2 aplicaciones por fila}
\end{figure}

\begin{center}

	\xmm{1} \xmmDoubleWordSmall{$P_1$}{$P_2$}{$P_3$}{$P_4$}

	\xmm{2} \xmmDoubleWordSmall{$P_5$}{$P_6$}{$P_7$}{$P_8$}

	\texttt{PMAXUB} \xmm{1}, \xmm{2} \hfill

	\xmm{1} \xmmDoubleWordSmall{\tiny$MAX(P_1,P_5)$}{\tiny$MAX(P_2,P_6)$}{\tiny$MAX(P_3,P_7)$}{\tiny$MAX(P_4,P_8)$}

	$MAX(PM,P)$ = \xmmDoubleWordSmall{\tiny$MAX(PM^r,P^r)$}{\tiny$MAX(PM^g,P^g)$}{\tiny$MAX(PM^b,P^b)$}{\tiny$MAX(PM^a,P^a)$}
\end{center}

Una vez realizado esto sobre cada fila, vamos a tener en el registro los cuatro posibles pixeles máximos. Los mismos son comparados entre sí para conseguir el pixel definitivo que contenga el máximo de cada componente. Para esto podemos seguir usando PMAXUB, duplicamos el registro y shifteamos 8 bytes a la derecha uno de ellos, para que nos quede desplazado 2 pixeles y podamos comparar el primer par de pixeles del registro contra el segundo par. Repetimos este proceso desplazando 4 bytes para comparar los últimos 2 pixeles. Una vez realizado esto nos va a quedar en la parte menos significativa del registro el pixel que estábamos buscando.

\begin{center}
	\xmm{1} \xmmDoubleWordSmall{$M_1$}{$M_2$}{$M_3$}{$M_4$}

	\xmm{2} $\leftarrow$ \xmm{1}

	\texttt{PSRLDQ} \xmm{2}, \texttt{8} \hfill

	\xmm{2} \xmmDoubleWordSmall{0}{0}{$M_1$}{$M_2$}

	\texttt{PMAXUB} \xmm{1}, \xmm{2} \hfill

	\xmm{1} \xmmDoubleWordSmall{$M_1$}{$M_2$}{\tiny$MAX(M_1,M_3)$}{\tiny$MAX(M_2,M_4)$}

	\xmm{2} $\leftarrow$ \xmm{1}

	\texttt{PSRLDQ} \xmm{2}, \texttt{4} \hfill

	\xmm{2} \xmmDoubleWordSmall{0}{$M_1$}{$M_2$}{\tiny$MAX(M_1,M_3)$}

	\texttt{PMAXUB} \xmm{1}, \xmm{2} \hfill

	\xmm{1}
	\vspace{0.1cm}
	\begin{tabular}{|C{1cm}|C{1cm}|C{1cm}|C{3.8cm}|}\hline
		X & X & X & $MAX(M_1,M_2,M_3,M_4)$ \\ \hline
	\end{tabular}
	\vspace{0.1cm}
\end{center}

Ya conseguido el máximo, ahora tenemos que realizar la combinación lineal entre el pixel original, el nuevo pixel que generamos y el parámetro. Lo que queremos es hacer la cuenta para cada componente en paralelo: para esto podemos usar instrucciones SIMD para multiplicar las componentes de los pixeles en paralelo por el parámetro. Como el mismo es un valor decimal (\texttt{float}), convertimos cada componente \texttt{float} quedándonos todas las componentes en un registro XMM. Para multiplicarlas por el parámetro, armamos un registro XMM que lo contenga 4 veces y aplicamos MULPS, quedando de resultado la multiplicación por cada componente contra el parámetro cada una en floats. Luego realizamos un proceso análogo con el pixel original y $1 - val$, y sumamos estos dos resultados. Convertimos este resultado de vuelta a enteros y lo guardamos en la imagen destino, continuamos con el resto de los pixeles.

\subsection{Análisis preliminar}

\begin{center}
	\includegraphics[scale=0.5]{img/maxCloser_CvsASMvsO3.png}
\end{center}

Una vez más, este filtro se comporta de manera lineal con respecto al tamaño de la imagen en pixeles, independientemente de la implementación usada. Para cada pixel deben procesarse 49 pixeles (por el kernel de 7x7), pero esta cantidad es fija, por lo cual es lógico asumir que el filtro sea lineal. Por otro lado, para el caso del margen blanco (que implica unicamente procesar el pixel original), la cantidad de pixeles abarcada es muy poca para que estos tengan un impacto visible en la performance.

También podemos ver claramente como el filtro implementado en ASM corre más rápido que en C, de hecho, al incrementar el tamaño de la imagen la diferencia es aún más notable. Si bien hay una amplia mejora al compilar con optimizaciones \texttt{-O3}, no llega a mejorar la versión de ASM.

\begin{center}
	\includegraphics[scale=0.5]{img/maxCloser_CvsASMvsO3_bars.png}
\end{center}

Analizando de cerca, podemos ver que la diferencia entre las distintas implementaciones es la mayor entre todos los filtros estudiados: la implementación en ASM corren en menos de 25\% del tiempo en promedio para una misma imagen.

Esto es un comportamiento esperable ya que en C estamos yendo a buscar el pixel a memoria por cada pixel del kernel, en cambio en ASM con las instrucciones SIMD cada 7 pixeles lo pedimos 2 veces. Además, en ASM aprovechamos y hacemos las cuentas para el pixel destino en paralelo.

\subsection{Experimantación}
Como reveló el análisis preliminar, la cantidad de pixeles origen a procesar por pixel destino tuvo un gran impacto en la performance de nuesto filtro. Decidimos ver cómo impacta en ambas implementaciones si agrandamos el tamaño de kernel. Para esto corrimos los mismos tests que antes, pero ahora con un kernel de 11x11 en vez que de 7x7.

Nuestra hipótesis fue que esta medida afectaría con más contundencia a la implementación de C, si bien en ASM va a tardar más, en relación al kernel mas chico, no le va afectar tanto. Esto es debido a que en C trae cada pixel del kernel uno por vez, y como estamos incrementando el tamaño del kernel en casi 2.5 veces, esperamos una performance peor en este orden. Pero en ASM aprovechando la paralelización de datos, levantamos 11 pixeles con 3 llamadas a memoria nada más.

% TODO: medir esto de nuevo, asegurarse de usar O3, conservar datos en jupyter

\begin{center} 
	\includegraphics[scale=0.5]{img/maxCloser_KERNEL.png}
	\includegraphics[scale=0.5]{img/maxCloser_KERNEL_ASM.png}
	\includegraphics[scale=0.5]{img/maxCloser_KERNEL_C.png}
\end{center}

Si observamos los gráficos, podemos ver como en C con el kernel de 11x11 tarda casi 2.5 veces más, como la relación de diferencia de pixeles en los kernels, y para ASM hay nada mas una diferencia de 1.2 y 1.4 para instancias más grandes. Como habíamos predicho en la hipótesis, este cambio en el tamaño del kernel afectó bastante más a la implementación de C que a la de ASM.

\subsubsection*{Análisis de las variables de entrada}

Queremos realizar un análisis sobre el tiempo en que tarda en terminar el filtro en base a los parámetros que recibe, distintos tipos de imagen, tamaños y el valor que afecta la combinación lineal. Nuestra hipótesis es que las pruebas que hacemos no van a tener un impacto significativo sobre el rendimiento del filtro, ya que no hay optimizaciones que hagan variar el tiempo final según los parámetros(si la cantidad de pixeles como vimos en el análisis preliminar).

Primero hagamos un análisis sobre distintas imágenes y cómo reacciona el algoritmo frente a ellas, en este caso vamos a comparar las imágenes de lena y colores (las de los tests de la cátedra), una de los siguientes colores azul, rojo, verde, gris, blanco y negro y por último una imagen generada aleatoriamente(en python), donde cada pixel es de un color elegido aleatoriamente distribuido uniformemente. Para la experimentación se dejó fijo en todas las imágenes el tamaño(512x512), el valor de la combinación final en 0.5, y se corrió la experimentación unas 100 veces y para cada color nos quedamos con el que menor tiempo resultó.

\begin{center} 
	\includegraphics[scale=0.5]{img/maxCloser_PARAM_IMG.png}
\end{center}

A simple vista se ve que varía muy poco, y posiblemente esto sea producto de la inexactitud de la experimentación. Este resultado se debe a que los cálculos se hacen de cualquier manera, siempre buscamos en el kernel el mayor de las componentes, una cosa que se nos ocurrió es que la imagen aleatoria podría haber llegado a tardar un poco más debido a que cambia más el valor del máximo en la búsqueda del kernel, no como en los colores fijos que no cambia, pero suponemos que esto no tiene impacto y que el procesador no pierde más tiempo cuando por ejemplo en la instrucción pmaxub tiene que cambiar los valores en los registros.

El siguiente experimento fue sobre el parámetro de entrada de punto flotante que se aplica sobre la combinación lineal, notemos que acá si podríamos haber optimizado cuando el valor viene en 0 ya que nos podemos ahorrar la búsqueda del máximo, pero es una optimización simple que no es a lo que apuntamos en este trabajo. Corrimos este experimento como en el anterior pero para las imágenes de lena y colores y variamos el valor entre los siguientes 0 0.313 0.5 0.713 1.

\begin{center} 
	\includegraphics[scale=0.5]{img/maxCloser_PARAM_VAL.png}
\end{center}

De nuevo podemos observar la poca varianza entre los distintos valores, el que menos ticks dió fue el de 0.5 con 12335487 ticks y el que mayor fue el de 1 con 12416301 ticks, la diferencia entre estos dos representa el $0,65\%$ de este último. Con estos resultados podemos afirmar que no hay diferencia significativa.

Por último probamos cómo afectan las diferentes proporciones de tamaño, básicamente queremos ver como afecta si la imagen es más ancha que alta, en este caso probamos para las imágenes de lena y colores,  con el valor fijo en 0.5 y probamos para tamaños de 512x128 y 128x512 ya que tienen la misma cantidad de pixeles.

\begin{center} 
	\includegraphics[scale=0.5]{img/maxCloser_PARAM_SIZE.png}
\end{center}

Esta vez vemos una leve mejora en el que es más ancho, 2975394 vs 2956806, el ancho corre en el $99.3\%$, la verdad que es un muy poca la diferencia, esto podría provenir de dibujar las primeras 3 filas y las ultimas son mas fácil de pintar porque no se calcula nada(las filas blancas) y si son más anchas, vamos a poner mas pixeles dentro de estas(la cantidad de pixeles blancos es la misma pero en el otro están concentrados en las primeras y últimas columnas). Con estos resultados, podemos afirmar que sabemos como se va a comportar el algoritmo, ya que los parametros (salvo la cantidad de pixeles) no nos afectan en el rendimiento.

\subsubsection*{Unrolling}

En este experimento, nuevamente aplicamos la técnica de unrolling de ciclos, sobre el kernel, cuando estamos ciclando las filas, ya que nuestro kernel tiene una cantidad constantes de filas, y gracias a esto podemos aplicar esta técnica. Pero no como en el experimento que ya aplicamos esta técnica, tenemos la hipótesis de que la mejora será muy leve o nula, esto lo suponemos ya que el grueso del cómputo del filtro está en las pegadas a memoria, y en comparación a los saltos condicionales del ciclo estos últimos no son relevantes. Corrimos los mismos tests que en el analisis preliminar para la version original y la version con unrolling, ademas agregamos un tamaño de más de 5120x5120.

\begin{center} 
	\includegraphics[scale=0.5]{img/maxCloser_Unroll_compare.png}
	\includegraphics[scale=0.5]{img/maxCloser_Unroll_small.png} %512x512
	\includegraphics[scale=0.5]{img/maxCloser_Unroll_big.png} %5120x5120
\end{center}

Hay una mejora, pero en comparación a lo que tarda el algoritmo es muy poca, para la imagen de 512x512 resulto en una mejora del $4\%$ y y para la imagen de 5120x5120 hay una mejora del $5\%$, que no es mucho pero si estamos en una imagen muy grande o tenemos que optimizar mucho podria llegar a valer la pena. 